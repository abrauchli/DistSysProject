\section{Android Application}
Our Android application is divided into different packages, each dedicated to a specific purpose. The separation follows the MVC model and provides a good base for the software's different parts to evolve more or less independantly from each other.

\todo{Andrea: update this}
\begin{description}
	\item[\code{ch.ethz.inf.vs.android.g54.a4.frontend}]\hfill\\This part consists of the layout.
	\item[\code{ch.ethz.inf.vs.android.g54.a4.net}]\hfill\\Here are all the classes concerning the communication with the server, providing an interface to the other packages for requesting information. The retrieved information is then translated into convenient objects, rather than returning just the JSON objects.
	\item[\code{ch.ethz.inf.vs.android.g54.a4.logic}]\hfill\\In here is all the application logic and controlling, which relies on services of the other packages.
\end{description}

\subsection{Network library and Lazy Objects}
\todo{TODO: Andy, Andrea}
The network library serves as abstraction of the server's functions. All communications happen transparently to the clients (View / Controller) through asynchronous calls, allowing non-blocking loading. The model classes describe the logical structuring of the server's information in the form of lazy objects. Thus a Building object holds a reference to a possibly not yet loaded Floor object. The state of an object can be queried through the isLoaded() method and should it be necessary load its contents through the loadAsync(handler) method. The android handler will be served a message when the object is ready. All these lazy objects inherit from the LazyObject class and share the loading and caching mechanisms.

\subsection{User Interface}
\todo{TODO: Steven}

\subsubsection{MapView}
\todo{TODO: Marc}
The MapView displays floor maps of ETH buildings and location markers. Latter are used to display the current position of the app client and the positon of WiFi access points. We considered many different implementations, but in the end rejected most of them. Our first try was to extend a WebView, because it natively can be scrolled and zoomed. The location markers would have been shown through web technologies such as HTML and JavaScript. The problem of this approach is to get positions of a click. That's why we have rejected it. Second try was a custom view with low-level touch events which worked reasonably. The location markers were directly drawn on the canvas and were clickable. But we didn't manage to get zooming working. The third and luckily last try was a total success. On stackoverflow.com\footnote{http://stackoverflow.com/a/2632722} we found a nice view displaying a picture which can be scrolled and zoomed. Aforementioned view is a extended ImageView which uses low-level touch events as well and matrices for the scrolling and zooming. Our MapView can display location markers which dynamically can be changed, updated and clicked. Furthermore the MapView can be centered in various ways including fitting in the view and centering on a specific map position. A nasty problem was hard to get rid of: OutOfMemoryExceptions due to the possibly large size of the used floor maps (e.g. 2505 * 1740 pixels for HG E) which are converted into bitmaps.
