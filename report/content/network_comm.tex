\section{Network Communication}
\todo{TODO: Andrea}
For the communication between server and client, we use REST with JSON. In this section we describe the API, that we defined for our system.

\subsection{Encapsulation}
We decided to encapsulate all the answers from the server in the same manner, such that we can easily handle error messages.\\
Every message has a field \code{ok} of type \code{boolean}, which indicates if the request succeeded.

\paragraph{Request Succeeded}
In this case the status is always \code{ok:true}. In addition to this field we have the following field:
\begin{description}
	\jsonField{result}{JSONObject}{result is specific to the kind of request that was sent to the server}
\end{description}

\paragraph{Request Failed}
In this case the status is always \code{ok:false}. In addition to this field we have the following field:
\begin{description}
	\jsonField{msg}{String}{message describing what kind of error ocurred}
\end{description}

\subsection{Campus Information}
The resource located at \code{<host>/c/} returns in it's \code{result} object an array of strings describing buildings for each Höngg and Zentrum.
\begin{description}
	\jsonField{Höngg}{JSONArray}{array of strings containing important buildings on campus Höngg}
	\jsonField{Zentrum}{JSONArray}{array of strings containing important buildings on campus Zentrum}
\end{description}

\subsection{Rooms and Buildings}

\paragraph{List of all Buildings}
The resource located at \code{<host>/r/} returns in it's \code{result} object a key-value list of all buildings belonging to ETH.
\begin{description}
	\jsonField{<building>}{JSONObject}{}
\end{description}
Every \code{JSONObject} belonging to a building has the following fields:
\begin{description}
	\jsonField{campus}{String}{can be ``Höngg'', ``Zentrum'' or ``other'' \todo{redundant}}
	\jsonField{name}{String}{name of this building (equal to the key)}
	\jsonField{address}{JSONObject}{see \todo{reference}}
\end{description}

\paragraph{Building detail}
The resource located at \code{<host>/r/<building>} returns in it's \code{result} object a detailed description of this building. It has the following fields:
\begin{description}
	\jsonField{name}{String}{name of this building (equal to \code{<building>} in the request)}
	\jsonField{address}{JSONObject}{see \todo{reference}}
	\jsonField{floors}{JSONObject}{key-value list of the floors in this building, see below}
\end{description}
Each floor identifier has a JSONObject attached to it, which has the following potential field:
\begin{description}
	\jsonField{map}{String}{URL to a map of this floor without any rooms marked, field omitted if there is no map available}
\end{description}

\paragraph{Floor detail}
The resource located at \code{<host>/r/<building>/<floor>} returns in it's \code{result} object a detailed description of this building. It has the following fields:
